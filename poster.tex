% --------------------------------------------------------------------------- %
% Poster for the ECCS 2011 Conference about Elementary Dynamic Networks.      %
% --------------------------------------------------------------------------- %
% Created with Brian Amberg's LaTeX Poster Template. Please refer for the     %
% attached README.md file for the details how to compile with `pdflatex`.     %
% --------------------------------------------------------------------------- %
% $LastChangedDate:: 2011-09-11 10:57:12 +0200 (V, 11 szept. 2011)          $ %
% $LastChangedRevision:: 128                                                $ %
% $LastChangedBy:: rlegendi                                                 $ %
% $Id:: poster.tex 128 2011-09-11 08:57:12Z rlegendi                        $ %
% --------------------------------------------------------------------------- %
\documentclass[a0paper,portrait]{baposter}

\usepackage[style=ieee]{biblatex}
\addbibresource{i-d.bib}
\addbibresource{report.bib}
\addbibresource{rfc.bib}

\usepackage{graphicx}


\newcommand{\irptitle}{Remaining Anonymous when using the Bitcoin Protocol}

%\newcommand\MYhyperrefoptions{bookmarks=true,bookmarksnumbered=true,
%pdfpagemode={UseOutlines},plainpages=false,pdfpagelabels=true,
%hidelinks,
%pdftitle={\irptitle},%<!CHANGE!
%pdfsubject={Bitcoin},%<!CHANGE!
%pdfauthor={Thomas A. Grainger},%<!CHANGE!
%pdfkeywords={Bitcoin, Anonymity, Cryptography, crypto-currencies, Peer to Peer}}

%\usepackage[\MYhyperrefoptions,pdftex]{hyperref}

\usepackage{relsize}    % For \smaller
\hyphenation{op-tical net-works semi-conduc-tor}
\usepackage{amssymb}
\usepackage{amsmath}
%%% Global Settings %%%%%%%%%%%%%%%%%%%%%%%%%%%%%%%%%%%%%%%%%%%%%%%%%%%%%%%%%%%


%%% Color Definitions %%%%%%%%%%%%%%%%%%%%%%%%%%%%%%%%%%%%%%%%%%%%%%%%%%%%%%%%%

\definecolor{bordercol}{RGB}{40,40,40}
\definecolor{headercol1}{RGB}{186,215,230}
\definecolor{headercol2}{RGB}{80,80,80}
\definecolor{headerfontcol}{RGB}{0,0,0}
\definecolor{boxcolor}{RGB}{186,215,230}

%%%%%%%%%%%%%%%%%%%%%%%%%%%%%%%%%%%%%%%%%%%%%%%%%%%%%%%%%%%%%%%%%%%%%%%%%%%%%%%%
%%% Utility functions %%%%%%%%%%%%%%%%%%%%%%%%%%%%%%%%%%%%%%%%%%%%%%%%%%%%%%%%%%

%%% Save space in lists. Use this after the opening of the list %%%%%%%%%%%%%%%%
\newcommand{\compresslist}{
  \setlength{\itemsep}{1pt}
  \setlength{\parskip}{0pt}
  \setlength{\parsep}{0pt}
}

%%%%%%%%%%%%%%%%%%%%%%%%%%%%%%%%%%%%%%%%%%%%%%%%%%%%%%%%%%%%%%%%%%%%%%%%%%%%%%%
%%% Document Start %%%%%%%%%%%%%%%%%%%%%%%%%%%%%%%%%%%%%%%%%%%%%%%%%%%%%%%%%%%%
%%%%%%%%%%%%%%%%%%%%%%%%%%%%%%%%%%%%%%%%%%%%%%%%%%%%%%%%%%%%%%%%%%%%%%%%%%%%%%%

\begin{document}

%%% Setting Background Image %%%%%%%%%%%%%%%%%%%%%%%%%%%%%%%%%%%%%%%%%%%%%%%%%%
\background{
  \begin{tikzpicture}[remember picture,overlay]%
  \draw (current page.north west)+(-2em,2em) node[anchor=north west]
  {\includegraphics[height=1.1\textheight]{pix/background}};
  \end{tikzpicture}
}

%%% General Poster Settings %%%%%%%%%%%%%%%%%%%%%%%%%%%%%%%%%%%%%%%%%%%%%%%%%%%
%%%%%% Eye Catcher, Title, Authors and University Images %%%%%%%%%%%%%%%%%%%%%%
\begin{poster}{
  grid=false,
  % Option is left on true though the eyecatcher is not used. The reason is
  % that we have a bit nicer looking title and author formatting in the headercol
  % this way
  eyecatcher=true, 
  borderColor=bordercol,
  headerColorOne=headercol1,
  headerColorTwo=headercol2,
  headerFontColor=headerfontcol,
  % Only simple background color used, no shading, so boxColorTwo isn't necessary
  boxColorOne=boxcolor,
  headershape=roundedright,
  headerfont=\Large\sf\bf,
  textborder=rectangle,
  background=user,
  headerborder=open,
  boxshade=plain
}
%%% Eye Cacther %%%%%%%%%%%%%%%%%%%%%%%%%%%%%%%%%%%%%%%%%%%%%%%%%%%%%%%%%%%%%%%
{
% The logos are compressed a bit into a simple box to make them smaller on the result
% (Wasn't able to find any bigger of them.)
  Electronics and\\
  Computer Science 
}
%%% Title %%%%%%%%%%%%%%%%%%%%%%%%%%%%%%%%%%%%%%%%%%%%%%%%%%%%%%%%%%%%%%%%%%%%%
{\sf\bf
  \irptitle
}
%%% Authors %%%%%%%%%%%%%%%%%%%%%%%%%%%%%%%%%%%%%%%%%%%%%%%%%%%%%%%%%%%%%%%%%%%
{
  \vspace{1em} Thomas Grainger\\
  {\smaller Email: t.grainger@ecs.soton.ac.uk}
}
%%% Logo %%%%%%%%%%%%%%%%%%%%%%%%%%%%%%%%%%%%%%%%%%%%%%%%%%%%%%%%%%%%%%%%%%%%%%
{
% The logos are compressed a bit into a simple box to make them smaller on the result
% (Wasn't able to find any bigger of them.)
  \includegraphics[height=3em]{img/University_of_Southampton}
}

\headerbox{Introduction}{name=introduction,column=0,row=0, span=3}{
While cryptographic digital currencies have previously been discussed as mathematical curiosities, Bitcoin\cite{satoshi}, a peer-to-peer electronic currency system has become the focus of significant interest and research effort.
This paper critically assess Bitcoin in the context of other digital currencies and investigates recent developments and discussions within the Bitcoin development community and scientific literature, with a particular focus on anonymity.
This paper also investigates methods used for de-anonymizing or revealing the identity of users of the Bitcoin network and how these methods can be contravened as well as proposing a novel automated solution to the issue of de-anonymizing techniques.

}

\headerbox{Existing Systems}{name=existing,column=0,below=introduction}{
\begin{itemize}
  \item Central Banking
  \item NetCash
  \item Chaumian e-Cash
  \item B-Money
\end{itemize}

}

\headerbox{References}{name=references,column=0,below=existing}{
\smaller                          % Make the whole text smaller
\vspace{-0.4em}                     % Save some space at the beginning
\renewcommand{\section}[2]{\vskip 0.05em}   % Omit "References" title
\printbibliography
}

\headerbox{Acknowledgements}{name=acknowledgements,column=0,below=references, above=bottom}{
\smaller            % Make the whole text smaller
\vspace{-0.4em}     % Save some space at the beginning
The author would like to thank Dr~Tim~Chown for his
support and supervision throughout this project, the mysterious Satoshi Nakamoto for the creation of the Bitcoin distributed currency system, and Gregory (gmaxwell) Maxwell and the other users of the freenode
\#bitcoin-dev Internet Relay Chat (IRC) channel.
} 



\headerbox{The Bitcoin Protocol}{name=bitcoin,span=2,column=1,row=0, below=introduction}{
The Bitcoin protocol and client software provides a concrete implementation of an altered version of the original b-money proposal, with defined transaction syntax such as dynamic payment scripts and the concept of outputs that must be spent in their entirety.  The only significant novel deviation is that \textcite{satoshi} has substituted the ``synchronous and unjammable anonymous broadcast channel''~\cite{b-money} with a decentralized time stamp server that is used to ratify the order in which transactions occurred.  This allows for a dynamically changing set of transaction authorizing servers to achieve a consensus of valid transactions. Any user can, on receipt of a transaction, query the time stamp server and determine whether a transaction has been invalidated by a prior transaction. This key difference is likely the reason that Bitcoin took crypto-currencies from mathematical curiosity to a usable system for storage and transfer of value.
\vspace{-0.2em}
\begin{center}
  \includegraphics[width=0.49\linewidth]{img/dyn/price}
  \includegraphics[width=0.49\linewidth]{img/dyn/speed-lin-ever}
\end{center}
}

\headerbox{Anonymization Techniques}
{name=anon,span=2,column=1,below=bitcoin}{

\begin{itemize}
  \item Multi-Input Transactions
  \item Change Addresses
\end{itemize}
}

\headerbox{De-Anonymization Techniques}
{name=deanon,span=2,column=1,below=anon,above=bottom}{
  \begin{itemize}
    \item Centralized Mixing
    \item ZeroCoin
    \item Automated In-Protocol Mixing
  \end{itemize}
}

\end{poster}
\end{document}
